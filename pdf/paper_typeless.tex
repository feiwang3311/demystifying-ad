%% ICFP: 27 pages + references

\documentclass[acmsmall,10pt,review,anonymous]{acmart}\settopmatter{printfolios=true,printccs=false,printacmref=false}
%% For double-blind review submission
%%\documentclass[acmsmall,10pt,review,anonymous]{acmart}\settopmatter{printfolios=true}
%% For single-blind review submission
%\documentclass[acmsmall,10pt,review]{acmart}\settopmatter{printfolios=true}
%% For final camera-ready submission
%\documentclass[acmsmall10pt,]{acmart}\settopmatter{}

%% Note: Authors migrating a paper from PACMPL format to traditional
%% SIGPLAN proceedings format should change 'acmsmall' to
%% 'sigplan'.


%% Some recommended packages.
% \usepackage{booktabs}   %% For formal tables:
%                         %% http://ctan.org/pkg/booktabs
% \usepackage{subcaption} %% For complex figures with subfigures/subcaptions
%                         %% http://ctan.org/pkg/subcaption



\usepackage[utf8]{inputenc}
\usepackage[T1]{fontenc}
\usepackage[scaled=0.8]{beramono}
\usepackage{amsmath}
\usepackage{amssymb}
\usepackage{booktabs}
\usepackage{xcolor,colortbl}
\usepackage{url}
\usepackage{listings}
\usepackage{paralist}
\usepackage{subcaption}
\usepackage{wrapfig}
\usepackage{enumitem}
\usepackage{multicol}
\usepackage{flushend}
\usepackage{bcprules}
\usepackage{nicefrac}
\usepackage{xfrac}
\usepackage[normalem]{ulem}

%\usepackage{paralist}
%\usepackage{titlesec}
%\usepackage{booktabs}
%\usepackage{parskip}

%\usepackage[x11names, rgb]{xcolor}
%\usepackage[utf8]{inputenc}
%\usepackage{tikz}
%\usetikzlibrary{snakes,arrows,shapes}
%\usepackage{amsmath}


%\usetikzlibrary{matrix,chains,scopes,positioning,arrows,fit}


% ----- begin macros/formatting

\input{../macros}
\input{../dot_macros}

%\setlength{\floatsep}{6pt plus 2pt minus 2pt}
%\setlength{\textfloatsep}{6pt plus 2pt minus 2pt}

%\setlength{\intextsep}{-10pt}
%\setlength{\intextsep}{0pt}
%
%\setlength{\floatsep}{2pt plus 2pt minus 2pt}        % space between floats
%\setlength{\textfloatsep}{4pt plus 2pt minus 2pt}    % space between floats and text
%\setlength{\intextsep}{2pt plus 2pt minus 2pt}
%
%\setlength{\dblfloatsep}{4pt plus 2pt minus 2pt}
%\setlength{\dbltextfloatsep}{2pt plus 2pt minus 2pt}

% ----- end macros/formatting


% ----- begin acm template info
%\raggedbottom
\makeatletter\if@ACM@journal\makeatother
%% Journal information (used by PACMPL format)
%% Supplied to authors by publisher for camera-ready submission
\acmJournal{PACMPL}
\acmVolume{2}
\acmNumber{ICFP}
\acmArticle{1}
\acmYear{2019}
\acmMonth{9}
\acmDOI{10.1145/nnnnnnn.nnnnnnn}
\startPage{1}
\else\makeatother
%% Conference information (used by SIGPLAN proceedings format)
%% Supplied to authors by publisher for camera-ready submission
\acmConference[PL'17]{ACM SIGPLAN Conference on Programming Languages}{January 01--03, 2017}{New York, NY, USA}
\acmYear{2017}
\acmISBN{978-x-xxxx-xxxx-x/YY/MM}
\acmDOI{10.1145/nnnnnnn.nnnnnnn}
\startPage{1}
\fi


%% Copyright information
%% Supplied to authors (based on authors' rights management selection;
%% see authors.acm.org) by publisher for camera-ready submission
\setcopyright{none}             %% For review submission
%\setcopyright{acmcopyright}
%\setcopyright{acmlicensed}
%\setcopyright{rightsretained}
%\copyrightyear{2017}           %% If different from \acmYear


%% Bibliography style
\bibliographystyle{ACM-Reference-Format}
%% Citation style
%% Note: author/year citations are required for papers published as an
%% issue of PACMPL.
\citestyle{acmauthoryear}   %% For author/year citations

% ----- end acm template info


\begin{document}


% ----- begin title and authors

\title[Demystifying Differentiable Programming: Shift/Reset the Penultimate Backpropagator]{Demystifying Differentiable Programming: \\Shift/Reset the Penultimate Backpropagator}

%\subtitle{(Functional Pearl)}

% Demystifying Differentiable Programming: Shift/Reset the Penultimate Backpropagator
% (lambda is of course still the ultimate, but we show that
% shift/reset are a useful preceding step, which translates to
% plain lambda via cps conversion)

% ----- end title and authors


% %% Author information
% %% Contents and number of authors suppressed with 'anonymous'.
% %% Each author should be introduced by \author, followed by
% %% \authornote (optional), \orcid (optional), \affiliation, and
% %% \email.
% %% An author may have multiple affiliations and/or emails; repeat the
% %% appropriate command.
% %% Many elements are not rendered, but should be provided for metadata
% %% extraction tools.

% %% Author with single affiliation.
% \author{First1 Last1}
% \authornote{with author1 note}          %% \authornote is optional;
%                                         %% can be repeated if necessary
% \orcid{nnnn-nnnn-nnnn-nnnn}             %% \orcid is optional
% \affiliation{
%   \position{Position1}
%   \department{Department1}              %% \department is recommended
%   \institution{Institution1}            %% \institution is required
%   \streetaddress{Street1 Address1}
%   \city{City1}
%   \state{State1}
%   \postcode{Post-Code1}
%   \country{Country1}
% }
% \email{first1.last1@inst1.edu}          %% \email is recommended

% %% Author with two affiliations and emails.
% \author{First2 Last2}
% \authornote{with author2 note}          %% \authornote is optional;
%                                         %% can be repeated if necessary
% \orcid{nnnn-nnnn-nnnn-nnnn}             %% \orcid is optional
% \affiliation{
%   \position{Position2a}
%   \department{Department2a}             %% \department is recommended
%   \institution{Institution2a}           %% \institution is required
%   \streetaddress{Street2a Address2a}
%   \city{City2a}
%   \state{State2a}
%   \postcode{Post-Code2a}
%   \country{Country2a}
% }
% \email{first2.last2@inst2a.com}         %% \email is recommended
% \affiliation{
%   \position{Position2b}
%   \department{Department2b}             %% \department is recommended
%   \institution{Institution2b}           %% \institution is required
%   \streetaddress{Street3b Address2b}
%   \city{City2b}
%   \state{State2b}
%   \postcode{Post-Code2b}
%   \country{Country2b}
% }
% \email{first2.last2@inst2b.org}         %% \email is recommended


% %% Paper note
% %% The \thanks command may be used to create a "paper note" ---
% %% similar to a title note or an author note, but not explicitly
% %% associated with a particular element.  It will appear immediately
% %% above the permission/copyright statement.
% \thanks{with paper note}                %% \thanks is optional
%                                         %% can be repeated if necesary
%                                         %% contents suppressed with 'anonymous'


% must come after author list ...

\lstMakeShortInline[keywordstyle=,%
              flexiblecolumns=false,%
              %basewidth={0.56em, 0.52em},%
              mathescape=false,%
              basicstyle=\tt]@




% \begin{abstract}
% \input{abstract}
% \end{abstract}


% 2012 ACM Computing Classification System (CSS) concepts
% Generate at 'http://dl.acm.org/ccs/ccs.cfm'.
\begin{CCSXML}
<ccs2012>
<concept>
<concept_id>10011007.10011006.10011008</concept_id>
<concept_desc>Software and its engineering~General programming languages</concept_desc>
<concept_significance>500</concept_significance>
</concept>
<concept>
<concept_id>10003456.10003457.10003521.10003525</concept_id>
<concept_desc>Social and professional topics~History of programming languages</concept_desc>
<concept_significance>300</concept_significance>
</concept>
</ccs2012>
\end{CCSXML}

\ccsdesc[500]{Software and its engineering~General programming languages}
\ccsdesc[300]{Social and professional topics~History of programming languages}
% End of generated code


%% Keywords
%% comma separated list
%\keywords{keyword1, keyword2, keyword3}  %% \keywords is optional


% \maketitle


%\category{D.3.4}{Programming Languages}{Processors -- Code generation, Optimization, Run-time environments}
%\category{D.1.3}{Programming Techniques}{Concurrent Programming -- Parallel programming}
%\category{D.3.3}{Programming Languages}{Language Constructs and Features -- Concurrent programming structures}

%\terms Design, Languages, Performance
%\keywords Staging, Code Generation, Data Structures, Extensible Compilers


% \input{10_intro}
% \input{20_main}
% \input{30_eval}
% \input{40_back}


% \clearpage
% {
%\linespread{0.80}
%\bibliographystyle{myabbrvnat}
%\bibliographystyle{abbrv}
%\setlength{\bibsep}{0pt}
%\renewcommand{\bibfont}{\footnotesize}
%\bibliographystyle{ACM-Reference-Format}
%\vspace{3cm}
% \bibliography{references}
% }

%\input{90_extra}

\newcommand{\Let}[2]{\texttt{let}\ #1\ \texttt{=}\ #2\ \texttt{in} \ }
\newcommand{\DLet}[2]{\underline{\texttt{let}}\ #1\ \texttt{=}\ #2\ \underline{\texttt{in}} \ }
\newcommand{\WLet}[2]{\uwave{\texttt{let}}\ #1\ \texttt{=}\ #2\ \uwave{\texttt{in}} \ }
\newcommand{\LetRec}[2]{\texttt{letrec}\ #1\ \texttt{=}\ #2\ \texttt{in} \ }
\newcommand{\DLetRec}[2]{\underline{\texttt{letrec}}\ #1\ \texttt{=}\ #2\ \underline{\texttt{in}} \ }
\newcommand{\Ref}[1]{\texttt{ref} \ #1}
\newcommand{\DRef}[1]{\underline{\texttt{ref}} \ #1}
\newcommand{\Fst}[1]{\texttt{fst} \ #1}
\newcommand{\DFst}[1]{\underline{\texttt{fst}} \ #1}
\newcommand{\Snd}[1]{\texttt{snd} \ #1}
\newcommand{\DSnd}[1]{\underline{\texttt{snd}} \ #1}

\newcommand{\Left}[1]{\texttt{inl} \ #1 }
\newcommand{\DLeft}[1]{\underline{\texttt{inl}} \ #1}
\newcommand{\Right}[1]{\texttt{inr} \ #1 }
\newcommand{\DRight}[1]{\underline{\texttt{inr}} \ #1 }

% \newcommand{\Left}[2]{\texttt{inl} \ #1 \ #2}
% \newcommand{\DLeft}[2]{\underline{\texttt{inl}} \ #1 \ #2}
% \newcommand{\Right}[2]{\texttt{inr} \ #1 \ #2}
% \newcommand{\DRight}[2]{\underline{\texttt{inr}} \ #1 \ #2}
\newcommand{\Pattern}[3]{\texttt{case} \ #1 \ \texttt{of} \ x^\tau \Rightarrow #2 \ \texttt{or} \ x^\tau \Rightarrow #3}
\newcommand{\Pat}[5]{\texttt{case} \ #1 \ \texttt{of} \ #2 \Rightarrow #3 \ \texttt{or} \ #4 \Rightarrow #5}
\newcommand{\DPat}[5]{\underline{\texttt{case}} \ #1 \ \underline{\texttt{of}} \ #2 \Rightarrow #3 \ \underline{\texttt{or}} \ #4 \Rightarrow #5}
\newcommand{\DPatL}[5]{\underline{\texttt{case}} \ #1 \\
                      &&\underline{\texttt{of}} \ #2 \Rightarrow #3 \\
                      &&\underline{\texttt{or}} \ #4 \Rightarrow #5}
\newcommand{\PatL}[5]{\texttt{case} \ #1 \\
                      &&\texttt{of} \ #2 \Rightarrow #3 \\
                      &&\texttt{or} \ #4 \Rightarrow #5}
\newcommand{\REF}[1]{\texttt{Ref} \ #1}
\newcommand{\DREF}[1]{\underline{\texttt{Ref}} \ #1}
\newcommand{\Dat}[0]{\underline{\at} \ }
\newcommand{\Sat}[0]{\overline{\at} \ }
\newcommand{\Wat}[0]{\uwave{\At} \ }
\newcommand{\At}[0]{\at \ }
\newcommand{\WLam}[0]{\uwave{\lambda\text{ }}}
\newcommand{\SLam}[0]{\overline{\lambda\text{ }}}
\newcommand{\DLam}[0]{\underline{\lambda\text{ }}}

\newcommand{\DAdd}[2]{\ #1 \ \underline{+} \ #2 \ }
\newcommand{\DMul}[2]{\ #1 \ \underline{*} \ #2 \ }
\newcommand{\DUpdate}[2]{#1 \ \underline{\mathrel{+}=} \ #2}
\newcommand{\DDref}[1]{\ \underline{!} \ #1}


\newcommand{\Shift}[2]{\texttt{shift} \ #1 \ \texttt{in} \ #2}
\newcommand{\SShift}[2]{\overline{\texttt{shift}} \ #1 \ \overline{\texttt{in}} \ #2}
\newcommand{\Res}[1]{\langle \ #1 \ \rangle}
\newcommand{\SRes}[1]{\overline \langle \ #1 \ \overline \rangle}
\newcommand{\DPair}[2]{\underline{(} #1 \ \underline{,} \ #2 \underline{)}}
\newcommand{\SPair}[2]{\overline{(} #1 \ \overline{,} \ #2 \overline{)}}

\newcommand{\UNIT}[0]{\texttt{Unit}}
\newcommand{\NUM}[0]{\RR \times (\REF \RR)}

\newcommand{\BOOL}[0]{\texttt{Boolean}}
\newcommand{\TR}[0]{\texttt{Tree}}
\newcommand{\TRR}[0]{\texttt{Tree'}}
\newcommand{\A}[0]{\texttt{A}}
\newcommand{\Leaf}[1]{\texttt{Leaf} \ #1}
\newcommand{\Branch}[2]{\texttt{Branch} \ #1 \ #2}


\newcommand{\Exp}[0]{\texttt{EXP}}
\newcommand{\Val}[0]{\texttt{VAL}}
\newcommand{\Id}[0]{\texttt{ID}}
\newcommand{\Env}[0]{\texttt{ENV}}

\newcommand{\Red}[1]{\textcolor[rgb]{0.00,0.00,1.00}{#1}} % do not use Red
% \newcommand{\Red}[1]{\textcolor[rgb]{1.00,0.00,0.00}{#1}}
\newcommand{\Green}[1]{\textcolor[rgb]{0.00,1.00,0.00}{#1}}
\newcommand{\Blue}[1]{\textcolor[rgb]{0.00,0.00,1.00}{#1}}
\newcommand{\ifThenElse}[3]{\texttt{if} \ #1 \ \texttt{then} \ #2 \ \texttt{else} \ #3}

\section*{Formalization}

\begin{figure}[h]
\vspace{-2ex}
\flushleft\footnotesize{
  We consider a variant of $\lambda$-calculus with let-bindings, products, mutable references, and injection-case.
  The language also contains delimited control operators shift and reset.
}
$$\footnotesize
\ba{rll}
\textbf{CORE LANGUAGE} \\
\textbf{Expressions:} && \Exp\\
% e &::=& c \ | \  x \  | \ e \ + \ e  \ |\  e \ * \ e \ | \ \lambda x. \ e \ | \ \At  e \ e \ | \ \texttt{let} \ x \ = \ e \ \texttt{in} \ e \\
%     &|& \texttt{fst} \ e \ | \ \texttt{snd} \ e \ | \ ( e, \ e ) \ | \ \texttt{ref} \ e \ | \ ! \ e \ | \ e := e \\
%     &|& \Left{e}{\tau} \ | \ \Right{\tau}{e} \ | \ \Pat{e}{x}{e}{x}{e} \\
% \textbf{Meta Language Expression:} && \Exp \\
e &::=& c \ | \  x \  | \ e \ + \ e  \ |\  e \ * \ e \ | \ \lambda x. \ e \ | \ \At e \ e \ | \ \texttt{let} \ x \ = \ e \ \texttt{in} \ e \\
    &|& \texttt{fst} \ e \ | \ \texttt{snd} \ e \ | \ ( e, \ e ) \ | \ \texttt{ref} \ e \ | \ ! \ e \ | \ e := e \\
    &|& \Left{e} \ | \ \Right{e} \ | \ \Pat{e}{x}{e}{x}{e} \\
    &|& \Shift{x}{e} \ | \ \Res{e} \\
% \textbf{Types:} \\
% \tau &::=& \texttt{Unit} \ | \ \RR \ | \ \texttt{Ref} \ \tau \ | \ \tau \rightarrow \tau \ | \ \tau \ \times \ \tau \ | \ \tau + \tau \ | \ \mu x.\tau \\
\textbf{Values:} && \Val\\
                 && \text{structure left abstract}\\
% v &::=& c \ | \ \lambda x. \ e \ | \ ( v, \ v ) \ | \ \Left{v}{\tau} \ | \ \Right{\tau}{v} \ | \ \texttt{loc} \ \\
% && \text{Note: loc denotes locations in implicit store}\\
\textbf{Environments:} && \Env = \Id \rightharpoonup \Val \\
\emptyset &::=& \lambda x. \bot \ \ \ \text{where $\bot$ is a non-terminating computation} \\
\rho[x \mapsto v] &::=& \lambda y. \ \ifThenElse{(x = y)}{v}{\At \rho \ y} \\
\\
\textbf{DERIVED CONSTRUCTS} \\
\textbf{Booleans and conditionals:} \\
% \text{Type: Boolean} &=& \UNIT + \UNIT \\
\text{Value: True}   &=& \Left{()}  \\
\text{Value: False}  &=& \Right{()} \\
\ifThenElse{b}{t}{e} &=& \Pat{b}{y}{t}{z}{e} \\
\textbf{Loops and recursion:} \\
\LetRec{f}{\lambda x. \ e_1} e_2 &=& \Let{f'}{\lambda f'. \lambda x. \ \Let{f}{\At f' \ f'} e_1}\\
                                  && \Let{f}{\At f' \ f'} e_2\\
% \Let{f}{\lambda f. \ \lambda x. \ e} \At (\ \At f \ f) \ x \\
                \text{Loops:} && \text{expressed as tail recursive functions}\\
\textbf{Tree data structures:} \\
% && \TR \ \A = \Leaf{\A} \ | \ \Branch{(\TR \ \A)}{(\TR \ \A)} \\% &=& \mu \alpha. \ \texttt{A} + (\alpha \times \alpha) \\
% \text{Example tree type: } \TR \ \texttt{A} &=& \mu \alpha. \ \texttt{A} + (\alpha \times \alpha) \\
\text{Example tree term: } t &=& \Right{(\Left{5}, \Left{6})}\\
% \text{Example tree term: } t^{\TR \ \RR} &=& \Right{\RR}{(\Left{5}{(\TR \ \RR) \times (\TR \ \RR)}, \Left{6}{(\TR \ \RR) \times (\TR \ \RR)})}\\
\\
\textbf{Syntactic sugar:} \\
                    % \hat y^\RR &=& (y, y')^{\NUM} \\ % \text{variable}\\
         % \hat y^{\tau \ne \RR} &=& y^{\ort\bra{\tau}} \\
    y_1 \ \mathrel{+}= \ ! \ y_2 &=& y_1 \ := \ ! \ y_1 \ + \ ! \ y_2 \\ % \text{reference} \\
    \Let{(y, y')}{e_1}{e_2} &=& \Let{\tilde y}{e_1} \Let{y}{\Fst \tilde y} \Let{y'}{\Snd \tilde y} e_2 \\ % \text{let with tuple} \\
    e_1 \ ; \ e_2 &=& \Let{\_}{e_1} e_2 \\ % \text{sequence} \\
    % \lambda (y, y'). e &=& \lambda \tilde y. \ \Let{y}{\Fst{\tilde y}} \Let{y'}{\Snd \tilde y} e \\
    % \Pat{e}{(y, y')}{e_1}{z}{e_2} &=& \Pat{e}{\tilde y}{\Let{y}{\Fst \tilde y} \Let{y'}{\Snd \tilde y} e_1}{z}{e_2}\\
    % \Pat{e}{y}{e_1}{(z, z')}{e_2} &=& \Pat{e}{y}{e_1}{\tilde z}{\Let{z}{\Fst \tilde z} \Let{z'}{\Snd \tilde z} e_2}\\
    % \Pat{e}{(y, y')}{e_1}{(z, z')}{e_2} &=& \PatL{e}{\tilde y}{\Let{y}{\Fst \tilde y} \Let{y'}{\Snd \tilde y} e_1}
    %                                                 {\tilde z}{\Let{z}{\Fst \tilde z} \Let{z'}{\Snd \tilde z} e_2}\\
\\
\ea
$$\vspace{2ex}
\footnotesize\flushleft{
  Above we show formal definitions of the language we consider. It serves as both object- and meta-language. 
  We show the syntax of the core languages (typeless, but types can be added),
  as well as derived constructs that express branches,
  loops, recursion, and recursive data structures in a standard way. 
  Syntactic sugar used in our presentation is also listed here.\\[1ex]

  We assume Barendregt's variable convention throughout, such that all bound variables are pairwise different 
  and different from the free variables.
  This allows several rules (labeled via \# in later figures) to be simplified compared to other formulations 
  (no need for variable substitution in transformation). \\[1ex]

  For transformation, we assume that the target language is the same as object language, unless noted otherwise.
}
\end{figure}
\clearpage

% $\footnotesize \textbf{Variable Sugaring:} $\\
% \vspace{-3ex}
% \begin{minipage}[t]{0.45\textwidth}
% $$\footnotesize
% \ba{rll}
% \text{forward mode:} \\
%                     \hat y^\RR &=& (y, y')^{\RR \times \RR} \\ % \text{variable}\\
%          \hat y^{\tau \ne \RR} &=& y^{\brt\bra{\tau}} \\
% \ea
% $$
% \end{minipage}\hspace{1ex}%
% \begin{minipage}[t]{0.45\textwidth}
% $$\footnotesize
% \ba{rll}
% \text{reverse mode:} \\
%                     \hat y^\RR &=& (y, y')^{\NUM} \\ % \text{variable}\\
%          \hat y^{\tau \ne \RR} &=& y^{\ort\bra{\tau}} \\
% \ea
% $$
% \end{minipage}\hspace{1ex}%


% $\footnotesize \textbf{Type Transformation:} $\\
% \vspace{-3ex}
% \begin{minipage}[t]{0.45\textwidth}
% $$\footnotesize
% \ba{rll}
% \text{forward mode:} \\
%     \brt\bra{\UNIT} &=& \UNIT\\
%     \brt\bra{\RR}               &=& \RR \times \RR\\
%     \brt\bra{\REF \tau} &=& \REF \brt\bra{\tau} \\
%     \brt\bra{\tau_1 \times \tau_2} &=& \brt\bra{\tau_1} \times \brt\bra{\tau_2} \\
%     \brt\bra{\tau_1 \rightarrow \tau_2} &=& \brt\bra{\tau_1} \rightarrow \brt\bra{\tau_2} \\
%     \brt\bra{\tau_1 + \tau_2} &=& \brt\bra{\tau_1} + \brt\bra{\tau_2} \\
%     \brt\bra{\mu y. \tau} &=& \mu y. \brt\bra{\tau} \\
%     \brt\bra{y} &=& y \\
% \ea
% $$
% \end{minipage}\hspace{1ex}%
% \begin{minipage}[t]{0.45\textwidth}
% $$\footnotesize
% \ba{rll}
% \text{reverse mode:} \\
%     \ort\bra{\UNIT} &=& \UNIT\\
%     \ort\bra{\RR} &=& \NUM\\
%     \ort\bra{\REF \tau} &=& \REF \ort\bra{\tau} \\
%     \ort\bra{\tau_1 \times \tau_2} &=& \ort\bra{\tau_1} \times \ort\bra{\tau_2} \\
%     \ort\bra{\tau_1 \rightarrow \tau_2} &=& \ort\bra{\tau_1} \rightarrow (\ort\bra{\tau_2} \rightarrow \UNIT) \rightarrow \UNIT \\
%     \ort\bra{\tau_1 + \tau_2} &=& \ort\bra{\tau_1} + \ort\bra{\tau_2} \\
%     \ort\bra{\mu y. \tau} &=& \mu y. \ort\bra{\tau} \\
%     \ort\bra{y} &=& y \\
% \ea
% $$
% \end{minipage}\hspace{1ex}%

% \end{figure}
% \clearpage

% standard interpretation and translation
\section{Standard Interpretation and Transformation}
% \subsection{Standard Metacircular Interpretation}
% \vspace{-2ex}
% \begin{figure}[h]
% $$\footnotesize
% \ba{rll}
% % \textbf{Standard Metacircular Interpretation:} \\
% \bra{.}: && \Exp \rightarrow \Env \rightarrow \Val\\
% \bra{ c }_\rho         &=& c\\
% \bra{ x }_\rho         &=& \At \rho \ x \\
% \bra{ e_1 + e_2 }_\rho &=& \bra{ e_1 }_\rho + \bra{ e_2 }_\rho \\
% \bra{ e_1 * e_2 }_\rho &=& \bra{ e_1 }_\rho * \bra{ e_2 }_\rho \\
% \bra{ \lambda y. \ e }_\rho &=& \lambda \tilde y. \ \bra{e}_{\rho[y \mapsto \tilde y]} \\
% \bra{ \At e_1 \ e_2}_\rho        &=& \At \bra{e_1}_\rho \ \bra{e_2}_\rho \\
% \bra{\Let{y}{e_1} e_2}_\rho &=& \Let{\tilde y}{\bra{e_1}_\rho} \bra{e_2}_{\rho[y \mapsto \tilde y]} \\
% \bra{\Fst{e}}_\rho               &=& \Fst{\bra{e}}_\rho \\
% \bra{\Snd{e}}_\rho               &=& \Snd{\bra{e}}_\rho \\
% \bra{\Ref{e}}_\rho               &=& \Ref{\bra{e}}_\rho \\
% \bra{! \ e}_\rho                 &=& ! \ {\bra{e}}_\rho \\
% \bra{e_1 := e_2}_\rho            &=& \bra{e_1}_\rho := \bra{e_2}_\rho \\
% \bra{(e_1, e_2)}_\rho            &=& (\bra{e_1}_\rho, \bra{e_2}_\rho) \\
% \bra{\Left{e}{\tau}}_\rho        &=& \Left{\bra{e}_\rho}{\tau} \\
% \bra{\Right{\tau}{e}}_\rho       &=& \Right{\tau}{\bra{e}_\rho} \\
% \bra{\Pat{e}{y_1}{e_1}{y_2}{e_2}}_\rho
% &=& \Pat{\bra{e}_\rho}{\tilde y_1}{\bra{e_1}_{\rho [y_1 \mapsto {\tilde y_1}]}}
%                       {\tilde y_2}{\bra{e_2}_{\rho [y_2 \mapsto {\tilde y_2}]}} \\
% \bra{\Shift{k}{e}}_\rho &=& \Shift{\tilde k}{\bra{e}_{\rho[k \mapsto \tilde k]}}\\
% \bra{\Res{e}}_\rho      &=& \Res{\bra{e}_\rho}\\
% \ea
% $$
% \end{figure}
% \vspace{-4ex}
% \flushleft
% \begin{footnotesize}
%   Here we show the standard metacircular interpretation of our formal language,
%   to be contrasted with interpretation for AD in later figures.
%   Since this interpretation is compositional, it can also be read as a translation.
%   We observe that environments always map $y \mapsto {\tilde y}$, hence we can drop
%   environments for the translation case and just use the name $y$ from the source
%   expression in the target expression and add a case $\bra{x} = x$ (shown below).
% \end{footnotesize}

\subsection{Standard Metacircular Transformation}
\vspace{-2ex}
\begin{figure}[h]
$$\footnotesize
\ba{rll}
% \textbf{Standard Metacircular Interpretation:} \\
\bra{.}: && \Exp \rightarrow \Exp \\
\bra{ c }         &=& c\\
\bra{ x }         &=& x \\
\bra{ e_1 + e_2 } &=& \bra{ e_1 } + \bra{ e_2 } \\
\bra{ e_1 * e_2 } &=& \bra{ e_1 } * \bra{ e_2 } \\
\bra{ \lambda y. \ e } &=& \lambda y. \ \bra{e} \\
\bra{ \At e_1 \ e_2}   &=& \At \bra{e_1} \ \bra{e_2} \\
\bra{\Let{y}{e_1} e_2} &=& \Let{y}{\bra{e_1}} \bra{e_2} \\
\bra{\Fst{e}}               &=& \Fst{\bra{e}} \\
\bra{\Snd{e}}               &=& \Snd{\bra{e}} \\
\bra{\Ref{e}}               &=& \Ref{\bra{e}} \\
\bra{! \ e}                 &=& ! \ {\bra{e}} \\
\bra{e_1 := e_2}            &=& \bra{e_1} := \bra{e_2} \\
\bra{(e_1, e_2)}            &=& (\bra{e_1}, \bra{e_2}) \\
\bra{\Left{e}}        &=& \Left{\bra{e}} \\
\bra{\Right{e}}       &=& \Right{\bra{e}} \\
% \bra{\Left{e}{\tau}}        &=& \Left{\bra{e}}{\tau} \\
% \bra{\Right{\tau}{e}}       &=& \Right{\tau}{\bra{e}} \\
\bra{\Pat{e}{y_1}{e_1}{y_2}{e_2}} &=& \Pat{\bra{e}}{y_1}{\bra{e_1}}{y_2}{\bra{e_2}} \\
\bra{\Shift{k}{e}} &=& \Shift{k}{\bra{e}}\\
\bra{\Res{e}}      &=& \Res{\bra{e}}\\
\ea
$$
\end{figure}

\flushleft
\begin{footnotesize}
Here we show the standard metacircular transformation using
% This metacircular transformation uses
shift/reset in the target language.
But what if we want to use a target language that does not provide shift/reset operators?
This can be achieved by moving the uses of shift/reset into the meta-language (so that they
are used at the time of translation), and generating
target terms in explicit CPS (without shift/reset).
\end{footnotesize}
\clearpage

\subsection{Standard CPS Transformation using shift/reset}
\vspace{-2ex}
\begin{figure}[h]
$$\footnotesize
\ba{rll}
% \textbf{Standard CPS Transformation using shift/reset:}\\
\bra{.}: && \Exp \rightarrow \Exp \\
\bra{ c }         &=& c \\
\bra{ x }         &=& x \\
\bra{ e_1 + e_2 } &=& \bra{e_1} \ \underline{+} \ \bra{e_2} \\
\bra{ e_1 * e_2 } &=& \bra{e_1} \ \underline{*} \ \bra{e_2} \\
\bra{ \lambda y. \ e } &=& \underline \lambda y. \underline \lambda k. \ \SRes{\Wat k \ \bra{e}} \\
\bra{ \ \At e_1 \ e_2} &=& \SShift{k}{\Dat (\Dat \bra{e_1} \bra{e_2})(\WLam a. \Sat k \ a)} \\
\#  \ \ \ \bra{\Let{y}{e_1} e_2} &=& \SShift{k}{\DLet{y}{\bra{e_1}}{\SRes{\Sat k \ \bra{e_2}}}}  \\
% \bra{\Let{y^\tau}{e_1} e_2} &=& \SShift{k}{\DLet{y_1^\tau}{\bra{e_1}}{\SRes{\Sat k \ \bra{e_2[y \leftarrow y_1]}}}} \\
\bra{\Fst{e}} &=& \DFst{\bra{e}} \\
\bra{\Snd{e}} &=& \DSnd{\bra{e}} \\
\bra{\Ref{e}} &=& \DRef{\bra{e}} \\
\bra{! \ {e}} &=& \underline{!} \ \bra{e} \\
\bra{e_1 := e_2} &=& \bra{e_1} \ \underline{:=} \ \bra{e_2} \\
\bra{(e_1, e_2)} &=& \DPair{\bra{e_1}}{\bra{e_2}} \\
\bra{\Left{e}}  &=& \DLeft{\bra{e}}  \\
\bra{\Right{e}} &=& \DRight{\bra{e}} \\
\bra{\Pat{e}{y_1}{e_1}{y_2}{e_2}} &=& \SShift{k}{\WLet{k_1}{\WLam a. \ \Sat k \ a}\\
                                   && \DPat{\bra{e}}{y_1}{\SRes{\Wat k_1 \ \bra{e_1}}}{y_2}{\SRes{\Wat k_1 \ \bra{e_2}}}} \\
\#  \ \ \ \bra{\Shift{k}{e}} &=& \SShift{k_1}{\DLet{k}{\underline \lambda v. \ \underline \lambda k_2. \ \Dat k_2 (\Sat k_1 \ v)}{\SRes{\bra{e}}}} \\
\bra{\Res{e}} &=& \SRes{\bra{e}} \\
\ea
$$
\end{figure}
\flushleft
\begin{footnotesize}
Here we show the standard CPS transformation using shift/reset.
The rules are adapted from~\citet{DBLP:journals/mscs/DanvyF92}, and the \# symbol denotes rules
that are simplified due to Barendregt's variable convention.
We also adopted the overline/underline notation from~\citet{DBLP:journals/mscs/DanvyF92},
such that overline denotes static/meta-language constructs,
and underline denotes dynamic/target-language constructs.
Departing slightly from~\citet{DBLP:journals/mscs/DanvyF92}, we introduce 
another \textbf{wavy underline notation} to implement proper tail calls.
Wavy underline denotes target-language terms just as normal underline, but 
wavy terms will be normalized with respect to the following contraction 
rules while the target expression is built up:
$$
\ba{rll}
\WLam y. \ \Wat e \ y &\rightarrow& e \\
\WLet{y}{y_1} e &\rightarrow& e [y \leftarrow y_1]\\
\ea
$$

\flushleft
Note that the wavy underline notation for let means that let-bindings should be
removed if and only if the RHS of the let-binding is just a variable (symbol).
This rule not strictly necessary for properly tail-recursive calls, but it removes
unnecessary symbol bindings for case expression in abstraction.

Note that we don't need to model ``code types'', because
we assume that object-level ASTs are modeled as proper tree types,
using recursive types, products, and sums.
\end{footnotesize}


\clearpage

% standard interpretation and translation
\subsection{Standard CPS Transformation}
\flushleft
\begin{footnotesize}
It is of course also possible to express the CPS transformation without shift/reset entirely by
switching the meta-languge code to CPS. This can be achieved formally by applying the same 
transformation as above to the meta-language translation code. The result is that shift/reset 
are fully erased from the right-hand sides of the translation.
\end{footnotesize}

\begin{figure}[h]
$$\footnotesize
\ba{rll}
% \textbf{Standard CPS Transformation:}\\
\bra{.}: && \Exp \rightarrow \Exp \\
\bra{ c }         &=& \overline \lambda \kappa.\ \Sat \kappa \ c\\
\bra{ x }         &=& \overline \lambda \kappa.\ \Sat \kappa \ x \\
\bra{ e_1 + e_2 } &=& \overline \lambda \kappa.\ \Sat \bra{e_1}(\overline \lambda y_1.\ \Sat \bra{e_2}(\overline \lambda y_2.\
                      \Sat \kappa (y_1 \ \underline{+} \ y_2))) \\
\bra{ e_1 * e_2 } &=& \overline \lambda \kappa.\ \Sat \bra{e_1}(\overline \lambda y_1.\ \Sat \bra{e_2}(\overline \lambda y_2.\
                      \Sat \kappa (y_1 \ \underline{*} \ y_2))) \\
\bra{ \lambda y. \ e } &=& \overline \lambda \kappa. \ \Sat \kappa \ (\underline \lambda y. \ \underline \lambda k.
                                                          \ \Sat \bra{e} \ (\SLam m. \Wat k \ m )) \\
\bra{ \ \At e_1 \ e_2}
&=& \overline \lambda \kappa. \ \Sat \bra{e_1}(\overline \lambda m. \ \Sat \bra{e_2}(\overline \lambda n. \
    \Dat (\ \Dat m \ n) \ (\WLam a. \ \Sat \kappa \ a))) \\
\#  \ \ \ \bra{\Let{y}{e_1} e_2}
&=& \overline \lambda \kappa. \ \Sat \bra{e_1}(\overline \lambda y_1. \ \DLet{y}{y_1} \Sat \bra{e_2} \ \kappa) \\
\bra{\Fst{e}} &=& \overline \lambda \kappa. \ \Sat \bra{e}(\overline \lambda y. \ \Sat \kappa \ (\DFst{y})) \\
\bra{\Snd{e}} &=& \overline \lambda \kappa. \ \Sat \bra{e}(\overline \lambda y. \ \Sat \kappa \ (\DSnd{y})) \\
\bra{\Ref{e}} &=& \overline \lambda \kappa. \ \Sat \bra{e}(\overline \lambda y. \ \Sat \kappa \ (\DRef{y})) \\
\bra{! \ {e}}         &=& \overline \lambda \kappa. \ \Sat \bra{e}(\overline \lambda y. \ \Sat \kappa \ (\underline ! \ {y})) \\
\bra{e_1 := e_2} &=& \overline \lambda \kappa. \ \Sat \bra{e_1}(\overline \lambda y_1. \ \Sat \bra{e_2}
   (\overline \lambda y_2. \ \Sat \kappa \ ( y_1 \ \underline{:=} \ y_2))) \\
\bra{(e_1, e_2)} &=& \overline \lambda \kappa. \ \Sat \bra{e_1}(\overline \lambda y_1. \ \Sat \bra{e_2}
   (\overline \lambda y_2. \ \Sat \kappa \ (\DPair{y_1}{y_2}))) \\
\bra{\Left{e}}  &=& \overline \lambda \kappa. \ \Sat \bra{e}(\overline \lambda y. \ \Sat \kappa \ (\DLeft{y})) \\
\bra{\Right{e}} &=& \overline \lambda \kappa. \ \Sat \bra{e}(\overline \lambda y. \ \Sat \kappa \ (\DRight{y})) \\
\bra{\Pat{e}{y_1}{e_1}{y_2}{e_2}}
&=& \overline \lambda \kappa. \ \WLet{k}{\WLam a. \ \Sat \kappa \ a} \Sat \bra{e}(\overline \lambda v. \\
                          && \DPat{v}{y_1}{\Sat \bra{e_1} (\SLam m. \ \Wat k \ m)}
                                     {y_2}{\Sat \bra{e_2} (\SLam n. \ \Wat k \ n)}) \\
\#  \ \ \ \bra{\Shift{k}{e}} &=& \overline \lambda \kappa. \ \DLet{k}{\underline \lambda a. \underline \lambda \kappa_1. \
     \Dat \kappa_1 (\Sat \kappa \ a)}{\Sat \bra{e}(\overline \lambda m. m)} \\
\bra{\Res{e}} &=& \overline \lambda \kappa. \Sat \kappa (\Sat \bra{e}(\overline \lambda m. m))\\
% \textbf{Properly tail-recursive:}\\
% \bra{ c }'         &=& \overline \lambda k.\ \Dat k \ c\\
% \bra{ x }'         &=& \overline \lambda k.\ \Dat k \ x \\
% \bra{ e_1 + e_2 }' &=& \overline \lambda k.\ \Sat \bra{e_1}(\overline \lambda y_1.\ \Sat \bra{e_2}(\overline \lambda y_2.\
%                       \Dat k (y_1 \ \underline{+} \ y_2))) \\
% \bra{ e_1 * e_2 }' &=& \overline \lambda k.\ \Sat \bra{e_1}(\overline \lambda y_1.\ \Sat \bra{e_2}(\overline \lambda y_2.\
%                       \Dat k (y_1 \ \underline{*} \ y_2))) \\
% \bra{ \lambda y^\tau. \ e }' &=& \overline \lambda k. \ \Dat k \ (\underline \lambda y^\tau. \ \underline \lambda k.
%                                                       \ \Sat \bra{e}' k) \\
% \bra{ \ \At e_1 \ e_2}'
% &=& \overline \lambda k. \ \Sat \bra{e_1}(\overline \lambda m. \ \Sat \bra{e_2}(\overline \lambda n. \
%     \Dat (\ \Dat m \ n) \ k)) \\
% \bra{\Let{y^\tau}{e_1} e_2}'
% &=& \overline \lambda k. \ \Sat \bra{e_1}(\overline \lambda y^\tau. \ \Sat \bra{e_2}' \ k) \\
% % &=& \overline \lambda k. \ \Sat \bra{e_1}(\overline \lambda y_1^\tau. \ \DLet{\tilde y}{y_1} \Sat \bra{e_2[y \leftarrow \tilde y]}' \ k) \\
% \bra{\Fst{e}}' &=& \overline \lambda k. \ \Sat \bra{e}(\overline \lambda y. \ \Dat k \ (\DFst{y})) \\
% \bra{\Snd{e}}' &=& \overline \lambda k. \ \Sat \bra{e}(\overline \lambda y. \ \Dat k \ (\DSnd{y})) \\
% \bra{\Ref{e}}' &=& \overline \lambda k. \ \Sat \bra{e}(\overline \lambda y. \ \Dat k \ (\DRef{y})) \\
% \bra{! \ {e}}' &=& \overline \lambda k. \ \Sat \bra{e}(\overline \lambda y. \ \Dat k \ (\underline ! \ {y})) \\
% \bra{e_1 := e_2}' &=& \overline \lambda k. \ \Sat \bra{e_1}(\overline \lambda y_1. \ \Sat \bra{e_2}
%    (\overline \lambda y_2. \ \Dat k \ ( y_1 \ \underline{:=} \ y_2))) \\
% \bra{(e_1, e_2)}' &=& \overline \lambda k. \ \Sat \bra{e_1}(\overline \lambda y_1. \ \Sat \bra{e_2}
%    (\overline \lambda y_2. \ \Dat k \ (\DPair{y_1}{y_2}))) \\
% \bra{\Left{e}{\tau}}'  &=& \overline \lambda k. \ \Sat \bra{e}(\overline \lambda y. \ \Dat k \ (\DLeft{y}{\tau})) \\
% \bra{\Right{e}{\tau}}' &=& \overline \lambda k. \ \Sat \bra{e}(\overline \lambda y. \ \Dat k \ (\DRight{\tau}{y})) \\
% \bra{\Pat{e_1}{y}{e_2}{z}{e_3}}'
% &=& \overline \lambda k. \ \Sat \bra{e_1}(\overline \lambda v. \ \DPatL{v}
%                              {y}{\Sat \bra{e_2}' k}
%                              {z}{\Sat \bra{e_3}' k}) \\
%                              % {y_1}{\DLet{\tilde y}{y_1} \Sat \bra{e_2[y \leftarrow \tilde y]}' k}
%                              % {z_1}{\DLet{\tilde z}{z_1} \Sat \bra{e_3[z \leftarrow \tilde z]}' k}) \\
% \bra{\Shift{k}{e}}' &=& \overline \lambda k. \ \DLet{k_1}{\underline \lambda a. \underline \lambda \kappa_1. \
%      \Dat \kappa_1 (\Dat k \ a)}{\Sat \bra{e[k \leftarrow k_1]}(\overline \lambda m. m)} \\
% \bra{\Res{e}}' &=& \overline \lambda k. \Dat k (\Sat \bra{e}(\overline \lambda m. m))\\
\ea
$$
\end{figure}
\vspace{-2ex}
\flushleft
\begin{footnotesize}
Here we show the standard CPS transformation without using shift/reset.
Similar to the figure before, rules and overline/underline notations are adapted from~\citet{DBLP:journals/mscs/DanvyF92}.
The \# symbol still denotes rules that are simplified due to Barendregt's variable convention,
and we use the same wavy underline notation to handle properly tail-recursive calls.
\end{footnotesize}
\vspace{2ex}


\clearpage
\section{Forward-mode AD}

\flushleft
\begin{footnotesize}
For AD (both forward-mode and reverse-mode), %interpretation and transformation),
we use the following variable sugaring with $\hat .$ notation. % and type transformation.
Note that the variable sugaring is not strictly necessary but we find it convenient
for $+$ and $*$ rules. Also note that this variable sugaring are always used
at position where we know for sure that the
sugared variables bind with Real typed values, so that they must have gradients
(denoted via variables with ').

% to highlight the type changes for bound variables. \\

Also note that for AD (both forward-mode and reverse-mode), we drop shift/reset terms
from the source language, since the focus is to provide a semantics for AD in a 
standard language, and shift/reset will play a crucial role for the semantics
of AD in reverse mode.

Note that our AD supports mutable states in the source language.
% \vspace{-2ex}
% \begin{multicols}{2}
$$
\ba{rll}
\textbf{Variable Sugaring:} \ \hat y &=& (y, y') \\ % \text{variable}\\
         %            \hat y^\RR &=& (y, y')^{\RR \times \RR} \\ % \text{variable}\\
         % \hat y^{\tau \ne \RR} &=& y^{\brt\bra{\tau}} \\
% \textbf{Type Transformation:}\\
%     \brt\bra{\UNIT} &=& \UNIT\\
%     \brt\bra{\RR}               &=& \RR \times \RR\\
% \ea
% $$
% \columnbreak
% $$
% \ba{rll}
% \\
%     \brt\bra{\REF \tau} &=& \REF \brt\bra{\tau} \\
%     \brt\bra{\tau_1 \times \tau_2} &=& \brt\bra{\tau_1} \times \brt\bra{\tau_2} \\
%     \brt\bra{\tau_1 \rightarrow \tau_2} &=& \brt\bra{\tau_1} \rightarrow \brt\bra{\tau_2} \\
%     \brt\bra{\tau_1 + \tau_2} &=& \brt\bra{\tau_1} + \brt\bra{\tau_2} \\
%     \brt\bra{\mu y. \tau} &=& \mu y. \brt\bra{\tau} \\
%     \brt\bra{y} &=& y \\
\ea
$$
% \end{multicols}
\end{footnotesize}

% \subsection{Forward-mode AD Interpretation}
% \begin{figure}[h]
% $$\footnotesize
% \ba{rll}
% % \textbf{Forward Mode AD Interpretation:} \\

% \texttt{Interpret}(f^{\RR \rightarrow \RR}) &=& \lambda x^\RR. \ \Let{\hat x}{(x, 1)} \\
%                        && \quad \quad \ \ \Let{\hat y}{\At \bro\bra{f}_\emptyset \ \hat x} \\
%                        && \quad \quad \ \ \ y' \\
% % \text{transformation:}\\
% \texttt{where} \ \bro \bra{.}: && \Exp \rightarrow \Env \rightarrow \Val \ \ \ \text{is defined as below:} \\
% \\
% \bro \bra{c^{\tau \neq \RR}}_\rho &=& c^\tau \\
% \bro \bra{c^{\RR}}_\rho           &=& \Red{(c, 0)^{\RR \times \RR}} \\
% \bro \bra{x}_\rho           &=& \At \rho \ x \\
% \bro \bra{y}_\rho            &=& \At \rho \ y \\

% \bro \bra{e_1 + e_2}_\rho &=&
%         \Blue{\Let{\hat y_1}{\bro\bra{e_1}_\rho}}\\
% \ \ \ &&\Blue{\Let{\hat y_2}{\bro\bra{e_2}_\rho}}\\
% \ \ \ &&\Blue{(y_1 + y_2, y_1' + y_2')} \\
% \bro \bra{e_1 * e_2}_\rho &=&
%         \Blue{\Let{\hat y_1}{\bro\bra{e_1}_\rho}} \\
% \ \ \ &&\Blue{\Let{\hat y_2}{\bro\bra{e_2}_\rho}} \\
% \ \ \ &&\Blue{(y_1 * y_2, y_1 * y_2' + y_1' * y_2)} \\
% \bro \bra{\lambda y. \ e}_\rho &=& \lambda \Red{\hat y_1}. \ \bro\bra{e}_{\rho[y \mapsto \hat y_1]} \\
% \bro \bra{\At e_1 \ e_2}_\rho &=& \At \bro \bra{e_1}_\rho \ \bro \bra{e_2}_\rho \\
% \bro \bra{\Let{y}{e_1} \ e_2}_\rho &=&
%  \Let{\Red{\hat y_1}}{\bro\bra{e_1}_\rho} \ \bro\bra{e_2}_{\rho[y \mapsto \hat y_1]} \\
% \bro\bra{\Fst e}_\rho &=& \Fst{\bro\bra e}_\rho \\
% \bro\bra{\Snd e}_\rho &=& \Snd{\bro\bra e}_\rho \\
% \bro\bra{\Ref e}_\rho &=& \Ref{\bro\bra e}_\rho \\
% \bro\bra{! \ e}_\rho  &=& \ ! \ \bro\bra{e}_\rho \\
% \bro\bra{e_1 := e_2}_\rho &=& \bro\bra{e_1}_\rho := \bro\bra{e_2}_\rho \\
% \bro\bra{(e_1, e_2)}_\rho &=& (\bro\bra{e_1}_\rho, \bro\bra{e_2}_\rho) \\
% \bro\bra{\Left{e}{\tau}}_\rho &=& \Left{\bro\bra{e}_\rho}{\brt\bra{\tau}} \\
% \bro\bra{\Right{\tau}{e}}_\rho&=& \Right{\brt\bra{\tau}}{\bro\bra{e}_\rho} \\
% \bro\bra{\Pat{e}{y}{e_1}{z}{e_2}}_\rho &=&
%     \Pat{\bro\bra{e}_\rho}{\Red{\hat y_1}}{\bro\bra{e_1}_{\rho[y \mapsto \hat y_1]}}
%                           {\Red{\hat z_1}}{\bro\bra{e_2}_{\rho[z \mapsto \hat z_1]}} \\
% \ea
% $$
% \end{figure}
% \flushleft
% \begin{footnotesize}
% Here we show the interpretation for forward-mode AD.
% Note the differences from standard interpretation, which are highlighted by color.
% Differences due to introduction of gradient field and type transformation are labeled in red.
% Differences due to gradient computation and let-bindings are labeled in blue.
% \end{footnotesize}
% \clearpage


\subsection{Forward-mode AD Transformation}
\begin{footnotesize}
\begin{figure}[h]
$$
\ba{rll}
% \textbf{Forward Mode AD Transformation:} \\
\texttt{Transform}(f) &=& \lambda x. \ \Let{\hat y}{\At \bro\bra{f} \ (x, 1)} \ y' \\
% \texttt{Transform}(f) &=& \lambda x. \ \Let{\hat x}{(x, 1)} \\
%                        && \quad \quad \ \ \Let{\hat y}{\At \bro\bra{f} \ \hat x} \\
%                        && \quad \quad \ \ \ y' \\
% \text{transformation:}\\
\texttt{where} \ \bro \bra{.}: && \Exp \rightarrow \Exp \ \ \ \text{is defined as below:} \\
\\
\bro \bra{c}.        &=& c  \ \ \text{if} \ c \notin \RR \\
\bro \bra{c=}        &=& \Red{(c, 0)} \ \ \text{if} \ c \in \RR \\
% \bro \bra{x}         &=& x \\
\bro \bra{y}         &=& y \\
\bro \bra{e_1 + e_2} &=&
        \Blue{\Let{\hat y_1}{\bro\bra{e_1}}}\\
\ \ \ &&\Blue{\Let{\hat y_2}{\bro\bra{e_2}}}\\
\ \ \ &&\Blue{(y_1 + y_2, y_1' + y_2')} \\
\bro \bra{e_1 * e_2} &=&
        \Blue{\Let{\hat y_1}{\bro\bra{e_1}}} \\
\ \ \ &&\Blue{\Let{\hat y_2}{\bro\bra{e_2}}} \\
\ \ \ &&\Blue{(y_1 * y_2, y_1 * y_2' + y_1' * y_2)} \\
\bro \bra{\lambda y. \ e} &=& \lambda y. \ \bro \bra{e} \\
\bro \bra{\At e_1 \ e_2} &=& \At \bro\bra{e_1} \ \bro\bra{e_2} \\
\bro \bra{\text{let} \ y = e_1 \ \text{in} \ e_2} &=&
 \Let{y}{\bro\bra{e_1}} \bro\bra{e_2} \\
\bro\bra{\Fst e} &=& \Fst{\bro\bra e} \\
\bro\bra{\Snd e} &=& \Snd{\bro\bra e} \\
\bro\bra{\Ref e} &=& \Ref{\bro\bra e} \\
\bro\bra{! \ e}  &=& \ ! \ \bro\bra{e} \\
\bro\bra{e_1 := e_2} &=& \bro\bra{e_1} := \bro\bra{e_2} \\
\bro\bra{(e_1, e_2)} &=& (\bro\bra{e_1}, \bro\bra{e_2}) \\
\bro\bra{\Left{e}} &=& \Left{\bro\bra{e}} \\
\bro\bra{\Right{e}}&=& \Right{\bro\bra{e}} \\
\bro\bra{\Pat{e}{y_1}{e_1}{y_2}{e_2}} &=&
    \Pat{\bro\bra{e}}{y_1}{\bro\bra{e_1}}{y_2}{\bro\bra{e_2}} \\
\ea
$$
\end{figure}
\vspace{-2ex}
\flushleft
Here we show the transformation for forward-mode AD.
% As in the standard metacircular interpretation and transformation, the only difference is in the use of environments.
Note that there is no metalanguage redex generated in the transformation,
so we elide underline denotations here, and by default let all constructs on the RHS be dynamic/target language constructs.
Rules that are different from standard transformation are highlighted by color.
% with red labeling differences due to introduction of gradient field,
% and blue labeling differences due to gradient computation and let-bindings.
\end{footnotesize}
\vspace{2ex}

\clearpage
\section{Reverse-mode AD}

% \flushleft
% \begin{footnotesize}
% For reverse-mode AD (both interpretation and transformation),
% we use the following variable sugaring with $\hat .$ notation and the following type transformation.
% Note that variable sugaring is not strictly necessary, but we find it helpful to highlight the type changes for bound variables.
% Note: type transformation for function types is valid for CPS, not shift/reset
% \vspace{-4ex}
% \begin{multicols}{2}
% $$\footnotesize
% \ba{rll}
% \textbf{Variable Sugaring:} \\
%                     \hat y^\RR &=& (y, y')^{\NUM} \\ % \text{variable}\\
%          \hat y^{\tau \ne \RR} &=& y^{\ort\bra{\tau}} \\
% \textbf{Type Transformation:}\\
%     \ort\bra{\UNIT} &=& \UNIT\\
%     \ort\bra{\RR} &=& \NUM\\
% \ea
% $$
% \columnbreak
% $$\footnotesize
% \ba{rll}
% \\
%     \ort\bra{\REF \tau} &=& \REF \ort\bra{\tau} \\
%     \ort\bra{\tau_1 \times \tau_2} &=& \ort\bra{\tau_1} \times \ort\bra{\tau_2} \\
%     \ort\bra{\tau_1 \rightarrow \tau_2} &=& \ort\bra{\tau_1} \rightarrow (\ort\bra{\tau_2} \rightarrow \UNIT) \rightarrow \UNIT \\
%     \ort\bra{\tau_1 + \tau_2} &=& \ort\bra{\tau_1} + \ort\bra{\tau_2} \\
%     \ort\bra{\mu y. \tau} &=& \mu y. \ort\bra{\tau} \\
%     \ort\bra{y} &=& y \\
% \ea
% $$
% \end{multicols}
% \end{footnotesize}
% \begin{minipage}[t]{0.45\textwidth}
% $$\footnotesize
% \ba{rll}
% \textbf{Variable Sugaring:} \\
%                     \hat y^\RR &=& (y, y')^{\NUM} \\ % \text{variable}\\
%          \hat y^{\tau \ne \RR} &=& y^{\ort\bra{\tau}} \\
% \ea
% $$
% \end{minipage}\hspace{1ex}%
% \begin{minipage}[t]{0.45\textwidth}
% $$\footnotesize
% \ba{rll}
% \textbf{Type Transformation:}\\
%     \ort\bra{\UNIT} &=& \UNIT\\
%     \ort\bra{\RR} &=& \NUM\\
%     \ort\bra{\REF \tau} &=& \REF \ort\bra{\tau} \\
%     \ort\bra{\tau_1 \times \tau_2} &=& \ort\bra{\tau_1} \times \ort\bra{\tau_2} \\
%     \ort\bra{\tau_1 \rightarrow \tau_2} &=& \ort\bra{\tau_1} \rightarrow (\ort\bra{\tau_2} \rightarrow \UNIT) \rightarrow \UNIT \\
%     \ort\bra{\tau_1 + \tau_2} &=& \ort\bra{\tau_1} + \ort\bra{\tau_2} \\
%     \ort\bra{\mu y. \tau} &=& \mu y. \ort\bra{\tau} \\
%     \ort\bra{y} &=& y \\
% \ea
% $$
% \end{minipage}\hspace{1ex}%
% \clearpage
% \vspace{-6ex}
% \subsection{Reverse-mode AD Interpretation using shift/reset}
% \vspace{-3ex}
% \begin{figure}[h]
% $$\footnotesize
% \ba{rll}
% % \textbf{Reverse Mode AD Interpretation}\\ \textbf{using shift/reset:} \\
% \texttt{Interpret}(f^{\RR \rightarrow \RR}) &=& \lambda x^\RR. \ \Let{\hat x}{(x, \Ref 0)} \\
%                        && \Res{\Snd{(\At \orb\bra{f}_{\emptyset}\ \hat x)} := 1.0 }; \\
%                        && ! \ x' \\
%      \texttt{where } \orb\bra{.}: && \Exp \rightarrow \Env \rightarrow \Val \ \ \ \text{is defined as below:} \\
%      \\
% \orb\bra{ c^{\tau \ne \RR} }_\rho &=& c\\
% \orb\bra{ c^{\RR} }_\rho          &=& \Red{(c, \Ref 0)} \\
% \orb\bra{ x }_\rho                &=& \At \rho \ x \\
% \orb\bra{ y }_\rho                &=& \At \rho \ y \\
% \orb\bra{ e_1 + e_2 }_\rho &=& \Blue{\Let{\hat y_1}{\orb\bra{e_1}_\rho}} \\
%                        && \Blue{\Let{\hat y_2}{\orb\bra{e_2}_\rho}} \\
%                        && \Blue{\texttt{shift } k \text{ in } \Let{\hat y}{(y_1 + y_2, \Ref 0)}} \\
%                        && \Blue{\quad \quad \quad \quad \quad k \ \hat y;} \\
%                        && \Blue{\quad \quad \quad \quad \quad y_1' \ \mathrel{+}= \ ! \ y';}\\
%                        && \Blue{\quad \quad \quad \quad \quad y_2' \ \mathrel{+}= \ ! \ y'}\\
% \orb\bra{ e_1 * e_2 }_\rho &=& \Blue{\Let{\hat y_1}{\orb\bra{e_1}_\rho}} \\
%                        && \Blue{\Let{\hat y_2}{\orb\bra{e_2}_\rho}} \\
%                        && \Blue{\texttt{shift } k \text{ in } \Let{\hat y}{(y_1 * y_2, \Ref 0)}} \\
%                        && \Blue{\quad \quad \quad \quad \quad k \ \hat y;} \\
%                        && \Blue{\quad \quad \quad \quad \quad y_1' \ \mathrel{+}= \ ! \ y' * y_2;}\\
%                        && \Blue{\quad \quad \quad \quad \quad y_2' \ \mathrel{+}= \ ! \ y' * y_1}\\
% \orb\bra{ \lambda y. \ e }_\rho &=& \lambda \Red{\hat y_1}. \ \orb\bra{e}_{\rho[y \mapsto \hat y_1]} \\
% \orb\bra{\At e_1 \ e_2}_\rho            &=& \At \orb\bra{e_1}_\rho \ \orb\bra{e_2}_\rho \\
% \orb\bra{\Let{y}{e_1} e_2}_\rho &=& \Let{\Red{\hat y_1}}{\orb\bra{e_1}_\rho} \orb\bra{e_2}_{\rho[y \mapsto \hat y_1]} \\
% \orb\bra{\Fst{e}}_\rho &=& \Fst{\orb\bra{e}}_\rho \\
% \orb\bra{\Snd{e}}_\rho &=& \Snd{\orb\bra{e}}_\rho \\
% \orb\bra{\Ref{e}}_\rho &=& \Ref{\orb\bra{e}}_\rho \\
% \orb\bra{! \ e}_\rho &=& ! \ {\orb\bra{e}}_\rho \\
% \orb\bra{e_1 := e_2}_\rho &=& \orb\bra{e_1}_\rho := \orb\bra{e_2}_\rho \\
% \orb\bra{(e_1, e_2)}_\rho &=& (\orb\bra{e_1}_\rho, \orb\bra{e_2}_\rho) \\
% \orb\bra{\Left{e}{\tau}}_\rho &=& \Left{\orb\bra{e}_\rho}{\ort\bra{\tau}} \\
% \orb\bra{\Right{\tau}{e}}_\rho &=& \Right{\ort\bra{\tau}}{\orb\bra{e}_\rho} \\
% \orb\bra{\Pat{e}{y}{e_1}{z}{e_2}}_\rho
% &=& \PatL{\orb\bra{e}_\rho}{\Red{\hat y_1}}{\orb\bra{e_1}_{\rho [y \mapsto {\hat y_1}]}}
%                            {\Red{\hat z_1}}{\orb\bra{e_2}_{\rho [z \mapsto {\hat z_1}]}} \\
% \ea
% $$
% \end{figure}
% \vspace{-3ex}
% \flushleft
% \begin{footnotesize}
%   Interpretation of reverse-mode AD with shift/reset and mutable state in the metalanguage.
%   Rules that are different from standard interpretation due to adding gradient field and type transformation
%   are labeled in red. Rules that are different from standard interpretation due to gradient computation are
%   labeled in blue.
% \end{footnotesize}
% \clearpage

\subsection{Reverse-mode AD Transformation using Target-Language shift/reset}
\vspace{-3ex}
\begin{figure}[h]
$$\footnotesize
\ba{rll}
% \textbf{Reverse Mode AD Interpretation}\\ \textbf{using shift/reset:} \\
\texttt{Transform}(f) &=& \lambda x. \ \Let{\hat x}{(x, \Ref 0)} \\
                       && \quad \ \ \ \Res{\Let{\hat z}{\At \orb\bra{f}\ \hat x} z' := 1.0}; \\
                       % && \Res{\Snd{(\At \orb\bra{f}\ \hat x)} := 1.0 }; \\
                       && \quad \ \ \ ! \ x' \\
     \texttt{where } \orb\bra{.}: && \Exp \rightarrow \Exp \ \ \ \text{is defined as below:} \\
     \\
\orb\bra{ c }         &=& c \ \ \text{if} \ c \notin \RR\\
\orb\bra{ c }         &=& \Red{(c, \Ref 0)}  \ \ \text{if} \ c \in \RR\\
% \orb\bra{ x }         &=& x\\
\orb\bra{ y }         &=& y \\
\orb\bra{ e_1 + e_2 } &=& \Blue{\Let{\hat y_1}{\orb\bra{e_1}}} \\
                       && \Blue{\Let{\hat y_2}{\orb\bra{e_2}}} \\
                       && \Blue{\texttt{shift } k \text{ in } \Let{\hat y}{(y_1 + y_2, \Ref 0)}} \\
                       && \Blue{\quad \quad \quad \quad \quad \At k \ \hat y;} \\
                       && \Blue{\quad \quad \quad \quad \quad y_1' \ \mathrel{+}= \ ! \ y';}\\
                       && \Blue{\quad \quad \quad \quad \quad y_2' \ \mathrel{+}= \ ! \ y'}\\
\orb\bra{ e_1 * e_2 } &=& \Blue{\Let{\hat y_1}{\orb\bra{e_1}}} \\
                       && \Blue{\Let{\hat y_2}{\orb\bra{e_2}}} \\
                       && \Blue{\texttt{shift } k \text{ in } \Let{\hat y}{(y_1 * y_2, \Ref 0)}} \\
                       && \Blue{\quad \quad \quad \quad \quad \At k \ \hat y;} \\
                       && \Blue{\quad \quad \quad \quad \quad y_1' \ \mathrel{+}= \ ! \ y' * y_2;}\\
                       && \Blue{\quad \quad \quad \quad \quad y_2' \ \mathrel{+}= \ ! \ y' * y_1}\\
\orb\bra{ \lambda y. \ e } &=& \lambda y. \ \orb\bra{e} \\
\orb\bra{\At e_1 \ e_2}            &=& \At \orb\bra{e_1} \ \orb\bra{e_2} \\
\orb\bra{\Let{y}{e_1} e_2} &=& \Let{y}{\orb\bra{e_1}} \orb\bra{e_2} \\
\orb\bra{\Fst{e}} &=& \Fst{\orb\bra{e}} \\
\orb\bra{\Snd{e}} &=& \Snd{\orb\bra{e}} \\
\orb\bra{\Ref{e}} &=& \Ref{\orb\bra{e}} \\
\orb\bra{! \ e} &=& ! \ {\orb\bra{e}} \\
\orb\bra{e_1 := e_2} &=& \orb\bra{e_1} := \orb\bra{e_2} \\
\orb\bra{(e_1, e_2)} &=& (\orb\bra{e_1}, \orb\bra{e_2}) \\
\orb\bra{\Left{e}} &=& \Left{\orb\bra{e}} \\
\orb\bra{\Right{e}} &=& \Right{\orb\bra{e}} \\
\orb\bra{\Pat{e}{y_1}{e_1}{y_2}{e_2}}
&=& \Pat{\orb\bra{e}}{y_1}{\orb\bra{e_1}}{y_2}{\orb\bra{e_2}} \\
\ea
$$
\end{figure}
\vspace{-3ex}
\flushleft
\begin{footnotesize}
  Transformation of reverse-mode AD with shift/reset and mutable state in the target language
  (identical to interpretation except for handling of environments).
  Rules that are different from standard transformation are highlighted by color.
  %  due to adding gradient field
  % are labeled in red. Rules that are different from standard transformation due to gradient computation are
  % labeled in blue.
\end{footnotesize}
\clearpage



\subsection{Reverse-mode AD Transformation using Meta-Language Shift/Reset}

\begin{figure}[h]
$$\footnotesize
\ba{rll}
% \textbf{Reverse Mode AD Transformation}\\
% \textbf{using Shift/Reset:} \\
\texttt{Transform}(f) &=& \underline \lambda x. \ \DLet{\hat x}{\DPair{x}{\DRef 0}} \\
                       % && \quad \ \ \ \SRes{\DLet{\hat z}{\Sat \orb\bra{f} \ \hat x}\ z' \ \underline{:=}\ 1.0} \underline{;} \\
                       && \quad \quad \Dat (\Dat \orb\bra{f} \ \hat x) \ (\WLam z. \ \DLet{\hat z}{z} z' \ \underline{:=}\ 1.0) \underline{;} \\
                       % && \quad \quad \ \ \SRes{\DSnd{(\Sat \orb\bra{f} \ \hat x)}\ \underline{:=}\ 1.0} \underline{;} \\
                       && \quad \ \ \ \DDref{x'} \ \\
          \texttt{where } \orb\bra{.}: && \Exp \rightarrow \Exp \ \ \ \text{is defined as below:} \\
                       \\
\orb\bra{ c }         &=& c \ \ \text{if} \ c \notin \RR \\
\orb\bra{ c }         &=& \Red{\DPair{c}{\DRef 0}} \ \ \text{if} \ c \in \RR \\
% \orb\bra{ x }         &=& x\\
\orb\bra{ y }         &=& y \\
\orb\bra{ e_1 + e_2 } &=& \Blue{\SShift{k}{\DLet{\hat y_1}{\orb\bra{e_1}}}} \\
                       && \quad \quad \quad \quad \ \ \ \Blue{\DLet{\hat y_2}{\orb\bra{e_2}}} \\
                       && \quad \quad \quad \quad \ \ \ \Blue{\DLet{\hat y}{\DPair{\DAdd{y_1}{y_2}}{\DRef 0}}} \\
                       && \quad \quad \quad \quad \ \ \ \Blue{\Sat k \ \hat y \underline{;}} \\
                       && \quad \quad \quad \quad \ \ \ \Blue{\DUpdate{y_1'}{\DDref{y'}}\underline{;}} \\
                       && \quad \quad \quad \quad \ \ \ \Blue{\DUpdate{y_2'}{\DDref{y'}}} \\
\orb\bra{ e_1 * e_2 } &=& \Blue{\SShift{k}{\DLet{\hat y_1}{\orb\bra{e_1}}}} \\
                       && \quad \quad \quad \quad \ \ \ \Blue{\DLet{\hat y_2}{\orb\bra{e_2}}} \\
                       && \quad \quad \quad \quad \ \ \ \Blue{\DLet{\hat y}{\DPair{\DMul{y_1}{y_2}}{\DRef 0}}} \\
                       && \quad \quad \quad \quad \ \ \ \Blue{\Sat k \ \hat y\underline{;}} \\
                       && \quad \quad \quad \quad \ \ \ \Blue{\DUpdate{y_1'}{\DMul{\DDref{y'}}{y_2}}\underline{;}} \\
                       && \quad \quad \quad \quad \ \ \ \Blue{\DUpdate{y_2'}{\DMul{\DDref{y'}}{y_1}}} \\
\orb\bra{ \lambda y. \ e } &=& \underline \lambda y. \ \underline \lambda k.
                                    \ \SRes{\Wat k \ \orb\bra{e}} \\
\orb\bra{ \ \At e_1 \ e_2} &=& \SShift{k}{\Dat (\Dat \orb\bra{e_1} \orb\bra{e_2})(\WLam a. \Sat k \ a)} \\
\#  \ \ \ \orb\bra{\Let{y}{e_1} e_2} &=& \SShift{k}{\DLet{y}{\orb\bra{e_1}}{\SRes{\Sat k \ \orb\bra{e_2}}}} \\
\orb\bra{\Fst{e}} &=& \DFst{\orb\bra{e}} \\
\orb\bra{\Snd{e}} &=& \DSnd{\orb\bra{e}} \\
\orb\bra{\Ref{e}} &=& \DRef{\orb\bra{e}} \\
\orb\bra{! \ {e}} &=& \DDref \orb\bra{e} \\
\orb\bra{e_1 := e_2} &=& \orb\bra{e_1} \ \underline{:=} \ \orb\bra{e_2} \\
\orb\bra{(e_1, e_2)} &=& \DPair{\orb\bra{e_1}}{\orb\bra{e_2}} \\
\orb\bra{\Left{e}}  &=& \DLeft{\orb\bra{e}}  \\
\orb\bra{\Right{e}} &=& \DRight{\orb\bra{e}} \\
\orb\bra{\Pat{e}{y_1}{e_1}{y_2}{e_2}} &=&
     \SShift{k}{\WLet{k_1}{\WLam a. \ \Sat k \ a}\\
  && \DPat{\orb\bra{e}}{y_1}{\SRes{\Wat k_1 \ \bra{e_1}}}
                       {y_2}{\SRes{\Wat k_1 \ \bra{e_2}}}} \\
\ea
$$
\end{figure}

\flushleft
\begin{footnotesize}
  Transformation of reverse-mode AD with metalanguage that contains shift/reset.
  Rules that are different from standard transformation are labeled by color.
  % with red highlighting differences due to introduction of gradient field,
  % and blue highlighting differences due to gradient computation.
\end{footnotesize}
\clearpage


\subsection{Reverse-mode AD Transformation in CPS}
\begin{figure}[h]
$$\footnotesize
\ba{rll}
% \textbf{Reverse Mode AD Transformation} \\
% \textbf{in CPS:} \\
\texttt{Transform}(f) &=& \underline \lambda x. \ \DLet{\hat x}{\DPair{x}{\DRef 0}} \\
                       % && \quad \quad \DLet{k}{\underline \lambda \ z. \ \DSnd{z} \ \underline{:=} \ 1.0} \\
                       % && \quad \quad \Sat \orb\bra{f} \ (\overline \lambda m. \ \Dat ( \Dat m \ \hat x) \ k)\underline{;} \\
                       && \quad \ \ \ \Sat \orb\bra{f} \ (\overline \lambda m. \ \Dat ( \Dat m \ \hat x) \ (\WLam z. \ \DLet{\hat z}{z} z' \ \underline{:=} \ 1.0))\underline{;} \\
                       && \quad \quad \DDref{x'} \ \\
          \texttt{where } \orb\bra{.}: && \Exp \rightarrow \Exp \ \ \ \text{is defined as below:} \\
                       \\
\orb\bra{ c }         &=& \overline \lambda \kappa.\ \Sat \kappa \ c \ \ \text{if} \ c \notin \RR \\
\orb\bra{ c }         &=& \overline \lambda \kappa.\ \Sat \kappa \ \Red{\DPair{c}{\DRef 0}} \ \ \text{if} \ c \in \RR \\
% \orb\bra{ x }         &=& \overline \lambda \kappa.\ \Sat \kappa \ x \\
\orb\bra{ y }         &=& \overline \lambda \kappa.\ \Sat \kappa \ y \\
\orb\bra{ e_1 + e_2 } &=& \overline \lambda \kappa. \
                         \Blue{\Sat \orb\bra{e_1}(\overline \lambda p_1. \ \Sat \orb\bra{e_2}(\overline \lambda p_2.} \ \ \#\# \\
                      && \quad \quad \Blue{\DLet{\hat y_1}{p_1}} \ \Blue{\DLet{\hat y_2}{p_2}} \\
                      && \quad \quad \Blue{\DLet{\hat y}{\DPair{\DAdd{y_1}{y_2}}{\DRef 0}}} \\
                      && \quad \quad \Blue{\Sat \kappa \ \hat y\underline{;}}\\
                      && \quad \quad \Blue{\DUpdate{y_1'}{\DDref y'}\underline{;}}\\
                      && \quad \quad \Blue{\DUpdate{y_2'}{\DDref y'}))}\\
\orb\bra{ e_1 * e_2 } &=& \overline \lambda \kappa. \
                         \Blue{\Sat \orb\bra{e_1}(\overline \lambda p_1. \ \Sat \orb\bra{e_2}(\overline \lambda p_2.} \ \ \#\# \\
                      && \quad \quad \Blue{\DLet{\hat y_1}{p_1}} \ \Blue{\DLet{\hat y_2}{p_2}} \\
                      && \quad \quad \Blue{\DLet{\hat y}{\DPair{\DMul{y_1}{y_2}}{\DRef 0}}} \\
                      && \quad \quad \Blue{\Sat \kappa \ \hat y\underline{;}} \\
                      && \quad \quad \Blue{\DUpdate{y_1'}{\DMul{\DDref y'}{y_2}}\underline{;}} \\
                      && \quad \quad \Blue{\DUpdate{y_2'}{\DMul{\DDref y'}{y_1}}))} \\
\orb\bra{ \lambda y. \ e } &=& \overline \lambda \kappa. \ \Sat \kappa \ (\underline \lambda y. \ \underline \lambda k.
                                  \ \Sat \orb\bra{e} (\SLam m. \ \Wat k \ m)) \\
\orb\bra{ \ \At e_1 \ e_2}
&=& \overline \lambda \kappa. \ \Sat \orb\bra{e_1}(\overline \lambda m. \ \Sat \orb\bra{e_2}(\overline \lambda n. \
    \Dat (\ \Dat m \ n) \ (\WLam a. \ \Sat \kappa \ a))) \\
\#  \ \ \ \orb\bra{\Let{y}{e_1} e_2}
&=& \overline \lambda \kappa. \ \Sat \orb\bra{e_1}(\overline \lambda y_1. \ \DLet{y}{y_1} \Sat \orb\bra{e_2} \ \kappa) \\
\orb\bra{\Fst{e}} &=& \overline \lambda \kappa. \ \Sat \orb\bra{e}(\overline \lambda y. \ \Sat \kappa \ (\DFst{y})) \\
\orb\bra{\Snd{e}} &=& \overline \lambda \kappa. \ \Sat \orb\bra{e}(\overline \lambda y. \ \Sat \kappa \ (\DSnd{y})) \\
\orb\bra{\Ref{e}} &=& \overline \lambda \kappa. \ \Sat \orb\bra{e}(\overline \lambda y. \ \Sat \kappa \ (\DRef{y})) \\
\orb\bra{! \ {e}} &=& \overline \lambda \kappa. \ \Sat \orb\bra{e}(\overline \lambda y. \ \Sat \kappa \ (\DDref{y})) \\
\orb\bra{e_1 := e_2} &=& \overline \lambda \kappa. \ \Sat \orb\bra{e_1}(\overline \lambda y_1. \ \Sat \orb\bra{e_2}
   (\overline \lambda y_2. \ \Sat \kappa \ ( y_1 \ \underline{:=} \ y_2))) \\
\orb\bra{(e_1, e_2)} &=& \overline \lambda \kappa. \ \Sat \orb\bra{e_1}(\overline \lambda y_1. \ \Sat \orb\bra{e_2}
   (\overline \lambda y_2. \ \Sat \kappa \ (\DPair{y_1}{y_2}))) \\
\orb\bra{\Left{e}}  &=& \overline \lambda \kappa. \ \Sat \orb\bra{e}(\overline \lambda y. \ \Sat \kappa \ (\DLeft{y})) \\
\orb\bra{\Right{e}} &=& \overline \lambda \kappa. \ \Sat \orb\bra{e}(\overline \lambda y. \ \Sat \kappa \ (\DRight{y})) \\
\orb\bra{\Pat{e}{y_1}{e_1}{y_2}{e_2}}
&=& \overline \lambda \kappa. \ \WLet{k}{\WLam a. \ \Sat \kappa \ a} \Sat \orb\bra{e}(\overline \lambda v. \
                                \DPatL{v}{y_1}{\Sat \orb\bra{e_1}(\SLam m. \ \Wat k \ m)}
                                         {y_2}{\Sat \orb\bra{e_2}(\SLam n. \ \Wat k \ n)}) \\
\\
\ea
$$
\end{figure}
\vspace{-4ex}
\flushleft
\begin{footnotesize}
  Transformation of source language for reverse-mode AD in CPS (meta language doesn't contain shift/reset).
  Rules that are different from standard CPS transformation are highlighted by color.
  % with red labeling differences due to introduction of gradient field,
  % and blue labeling differences due to gradient computation.\\
  Note that in the plus rule and the multiplication rule (labeled by \#\#),
  we avoided using variable sugaring in $\overline \lambda p_1$ and $\overline \lambda p_2$
  so that we can introduce dynamic let-binding for them. The dynamic let-bindings are necessary
  to preserve sharing, evaluation order, and asymptotic complexity, since the RHS of them are accessed multiple times
  via $y_1, y_1', y_2$, and $y_2'$.
  % although the $\overline \lambda \hat y_1$ and $\overline \lambda \hat y_2$ are static abstraction,
  % the let-binding from de-sugaring them will be dynamic (i.e. retained in the transformed code).
\end{footnotesize}

\clearpage


% \begin{figure}[h]
% $$\footnotesize
% \ba{rll}
% \textbf{Reverse Mode AD Transformation:} \\
% \text{properly tail recursive} \\
% \orb\bra{ c^{\tau \ne \RR} }' &=& \overline \lambda k.\ \Dat k \ c\\
% \orb\bra{ c^{\RR} }'          &=& \overline \lambda k.\ \Dat k \ (c, \Ref 0) \\
% \orb\bra{ x }'                &=& \overline \lambda k.\ \Dat k \ \hat x \\
% \orb\bra{ y }'                &=& \overline \lambda k.\ \Dat k \ \hat y \\
% \orb\bra{ e_1 + e_2 }' &=& \overline \lambda k. \ \Sat \orb\bra{e_1}(\overline \lambda \hat y_1. \ \Sat \orb\bra{e_2}(\overline \lambda \hat y_2. \\
%                        && \quad \quad \Let{\hat y}{(y_1 + y_2, \Ref 0)} \\
%                        && \quad \quad \Dat k \ \hat y;\\
%                        && \quad \quad y_1' \ \mathrel{+}= \ !\ y';\\
%                        && \quad \quad y_2' \ \mathrel{+}= \ !\ y'))\\
% \orb\bra{ e_1 * e_2 }' &=& \overline \lambda k. \ \Sat \orb\bra{e_1}(\overline \lambda \hat y_1. \ \Sat \orb\bra{e_2}(\overline \lambda \hat y_2. \\
%                        && \quad \quad \Let{\hat y}{(y_1 + y_2, \Ref 0)} \\
%                        && \quad \quad \Dat k \ \hat y;\\
%                        && \quad \quad y_1' \ \mathrel{+}= \ !\ y' * y_2;\\
%                        && \quad \quad y_2' \ \mathrel{+}= \ !\ y' * y_1))\\
% \orb\bra{ \lambda y^\tau. \ e }' &=& \overline \lambda k. \ \Dat k \ (\underline \lambda \hat y^\tau. \ \underline \lambda k.
%                                                           \ \Sat \orb\bra{e}' k) \\
% \orb\bra{ \ \At e_1 \ e_2}'
% &=& \overline \lambda k. \ \Sat \orb\bra{e_1}(\overline \lambda m. \ \Sat \orb\bra{e_2}(\overline \lambda n. \
%     \Dat (\ \Dat m \ n) \ k)) \\
% \orb\bra{\Let{y^\tau}{e_1} e_2}'
% &=& \overline \lambda k. \ \Sat \orb\bra{e_1}(\overline \lambda \hat y^\tau. \ \Sat \orb\bra{e_2} \ k) \\
% % &=& \overline \lambda k. \ \Sat \orb\bra{e_1}(\overline \lambda \hat y^\tau. \ \DLet{\tilde y}{y_1} \Sat \orb\bra{e_2[y \leftarrow \tilde y]} \ k) \\
% \orb\bra{\Fst{e}}'    &=& \overline \lambda k. \ \Sat \orb\bra{e}(\overline \lambda y. \ \Dat k \ (\DFst{y})) \\
% \orb\bra{\Snd{e}}'    &=& \overline \lambda k. \ \Sat \orb\bra{e}(\overline \lambda y. \ \Dat k \ (\DSnd{y})) \\
% \orb\bra{\Ref{e}}'    &=& \overline \lambda k. \ \Sat \orb\bra{e}(\overline \lambda y. \ \Dat k \ (\DRef{y})) \\
% \orb\bra{! \ {e}}'    &=& \overline \lambda k. \ \Sat \orb\bra{e}(\overline \lambda y. \ \Dat k \ (\underline ! \ {y})) \\
% \orb\bra{e_1 := e_2}' &=& \overline \lambda k. \ \Sat \orb\bra{e_1}(\overline \lambda y_1. \ \Sat \orb\bra{e_2}
%    (\overline \lambda y_2. \ \Dat k \ ( y_1 \ \underline{:=} \ y_2))) \\
% \orb\bra{(e_1, e_2)}' &=& \overline \lambda k. \ \Sat \orb\bra{e_1}(\overline \lambda y_1. \ \Sat \orb\bra{e_2}
%    (\overline \lambda y_2. \ \Dat k \ (\DPair{y_1}{y_2}))) \\
% \orb\bra{\Left{e}{\tau}}'  &=& \overline \lambda k. \ \Sat \orb\bra{e}(\overline \lambda y. \ \Dat k \ (\DLeft{y}{\ort\bra{\tau}})) \\
% \orb\bra{\Right{\tau}{e}}' &=& \overline \lambda k. \ \Sat \orb\bra{e}(\overline \lambda y. \ \Dat k \ (\DRight{\ort\bra{\tau}}{y})) \\
% % \orb\bra{\Pat{e_1}{y}{e_2}{z}{e_3}}'
% % &=& \overline \lambda k. \ \Sat \orb\bra{e_1}(\overline \lambda v. \ \DPatL{v}
% %                           {\hat y_1}{\DLet{\tilde y}{\hat y_1} \Sat \bra{e_2[y \leftarrow \tilde y]}' k}
% %                           {\hat z_1}{\DLet{\tilde z}{\hat z_1} \Sat \bra{e_3[z \leftarrow \tilde z]}' k}) \\
% \orb\bra{\Pat{e}{y_1}{e_1}{y_2}{e_2}}'
% &=& \overline \lambda k. \ \Sat \orb\bra{e}(\overline \lambda v. \ \DPatL{v}
%                           {\hat y_1}{\Sat \bra{e_1}' k}
%                           {\hat y_2}{\Sat \bra{e_2}' k}) \\
% \\
% \\
% \ea
% $$
% \end{figure}
% \clearpage

\newcommand{\LIST}[0]{\texttt{List}}
\newcommand{\LISTT}[0]{\texttt{List'}}
\newcommand{\LEN}[0]{\texttt{Nat}}

\section{Examples}
\flushleft
\begin{footnotesize}
  Below we show examples of CPS transformation for reverse-mode AD.
  The results of transformation are only composed of target language expressions (all metalanguage redex have been removed).
  For clarity we elide the underline notations. We also drop the \At symbol in application for readability.
\end{footnotesize}

$$\footnotesize
\ba{rll}
\textbf{Loop Example:} && \text{the function $f$ computes power($x$, $l$) in a loop} \\
% \text{This example shows power(x, n) in a loop. }\\
% \text{Type: } \LEN \ &=& \mu \alpha. \UNIT + \alpha \\
\text{Term: }f &=& \lambda x. \ \Let{l}{\Right{(\Right{(\Left{()})})}} \\
               && \Let{f_0}{\lambda f. \lambda ll. \lambda ac. \ \PatL{ll}
               {y_1}{ac \quad \quad \quad \quad \quad \quad \quad \quad \ \ \text{// if $l$ is $0$, return $ac$}}
               {y_2}{f \ f \ y_2 \ (x_0 * ac)}} \quad \quad \text{// if $l$ is not $0$, recurse on $x_0 * ac$ and $l-1$}
               \\
               &&  f_0 \ f_0 \ l \ 1 \\
% \text{Type of } f_0 &=& \mu \alpha. \alpha \rightarrow \LEN \rightarrow \RR \rightarrow \RR \rightarrow \RR
% \ \ \text{this type is often not allowed} \\s
\text{Transformation: } \\
\texttt{Transform}(f) &=& \lambda x. \ \Let{\hat x}{(x, \Ref 0)} \ (\lambda \tilde x. \ (\lambda k. \\
 && \quad \quad \Let{l}{\Right{(\Right{(\Left{()})})}} \\
 && \quad \quad \Let{f_0}{\lambda f. \lambda k_1. \  k_1 \ (\lambda ll. \lambda k_2. \  k_2 \ (\lambda {ac}. \lambda k_3. \\
 && \quad \quad \Pat{ll}{y_1}{ k_3 \ {ac}}{y_2}{\\
 && \quad \quad \quad \quad  (f \ f)(\lambda a_3. \  ( a_3 \ y_2) \ (\lambda a_2. \\
 && \quad \quad \quad \quad \Let{\hat z_1}{ x} \\
 && \quad \quad \quad \quad \Let{\hat z_2}{ {ac}} \\
 && \quad \quad \quad \quad \Let{\hat z}{(z_1 * z_2, \Ref 0)} \\
 && \quad \quad \quad \quad \quad (a_2 \ \hat z) k_3; \\
 && \quad \quad \quad \quad \quad z_1' \mathrel{+}= \ ! \ z' * z_2; \\
 && \quad \quad \quad \quad \quad z_2' \mathrel{+}= \ ! \ z' * z_1 )))}))} \\
 && \quad \quad (f_0 \ f_0) (\lambda a_{22}. \ (a_{22} \ l) \ (\lambda a_{11}.\ ( a_{11} \ (1, \Ref 0)) \ k))) \\
 && \quad \quad \hat x \ (\lambda y. \Let{\hat y}{y} y' := 1.0); \\
 && \quad \quad ! \ x' \\
\ea
$$



$$\footnotesize
\ba{rll}
\textbf{List Example:} && \text{the function $f$ computes the product of all reals in list $l$ with $x$ recursively} \\
% \text{Type: } \LIST \ &=& \mu \alpha. \UNIT + \RR \times \alpha \\
% \text{Type: } \LISTT \ &=& \ort\bra{\LIST} = \mu \alpha. \UNIT + (\RR \times \REF \RR) \times \alpha \\
\text{Term: }f &=& \lambda x. \ \Let{l}{\Right{(5, \Right{(6, \Left{()})})}} \\
               && \Let{f_0}{\lambda f. \lambda ll. \ \PatL{ll}
               {y_1}{x \quad \quad \quad \quad \quad \quad \quad \quad \quad \ \ \ \quad \quad \quad \quad \quad \text{// if $l$ is empty, return $x$}}
               {y_2}{(\Fst y_2) * ((f\ f) (\Snd y_2))}} \quad \quad \text{// if $l$ is not empty, return $l.$head$ * f(l.$tail$)$}
               \\
               &&  ( f_0 \ f_0) \ l \\
% \text{Type of } f_0 &=& \mu \alpha. \alpha \rightarrow \LIST \rightarrow \RR \rightarrow \RR \\
\text{Transformation: } \\
\texttt{Transform}(f) &=& \lambda x. \ \Let{\hat x}{(x, \Ref 0)} \ (\lambda \tilde x. \ (\lambda k. \\
 && \quad \quad \Let{l}{\Right{((5, \Ref 0), \Right{((6, \Ref 0), \Left{()})})}} \\
 && \quad \quad \Let{f_0}{\lambda  f. \lambda k_1. \  k_1 \ (\lambda ll. \lambda k_2. \\
 && \quad \quad \Pat{ll}{y_1}{k_2 \ \tilde x}{y_2}{\\
 && \quad \quad \quad \quad (f \ f)(\lambda a_1. \ (a_1 \ (\Snd y_2)) \ (\lambda a. \\
 && \quad \quad \quad \quad \Let{\hat z_1}{\Fst y_2} \\
 && \quad \quad \quad \quad \Let{\hat z_2}{a} \\
 && \quad \quad \quad \quad \Let{\hat z}{(z_1 * z_2, \Ref 0)} \\
 && \quad \quad \quad \quad \quad k_2 \ \hat z; \\
 && \quad \quad \quad \quad \quad z_1' \mathrel{+}= \ ! \ z' * z_2; \\
 && \quad \quad \quad \quad \quad z_2' \mathrel{+}= \ ! \ z' * z_1 ))})} \\
 && \quad \quad (f_0 \ f_0) (\lambda a_{11}. (a_{11} \ l) \ k))) \\
 && \quad \quad \hat x \ (\lambda y. \Let{\hat y}{y} y' := 1.0); \\
 && \quad \quad ! \ x' \\
\ea
$$

\newcommand{\RIGHT}[0]{\texttt{Right}}

$$\footnotesize
\ba{rll}
\textbf{Tree Example:} && \text{the function $f$ compute the sum of all reals in the binary tree $t$, with $x$ in each leaf node} \\
% \text{Type: } \TR  &=& \mu \alpha. \UNIT + (\RR \times (\alpha \times \alpha))\\
% \text{Type: } \TRR &=& \ort\bra{\TR} = \mu \alpha. \UNIT + ((\NUM) \times (\alpha \times \alpha))\\
% \text{Type: } \RIGHT &=& (\NUM) \times (\TRR \times \TRR)\\
\text{Term: } f
 &=& \lambda x. \ \Let{t}{\Right{(5, (\Left{()},\Left{()}))}} \\
  && \Let{f_0}{\lambda f. \lambda tt. \ \PatL{tt}
     {y_1}{x \quad \quad \quad \quad \quad \quad \quad \ \ \ \quad \quad \quad \text{// if $t$ is empty, return x}}
     {y_2}{(\Fst y_2) \ + \\
  && \quad \quad \ \ \ \quad  (f\ f\ (\Fst (\Snd y_2))) \ + \\
  && \quad \quad \ \ \ \quad  (f\ f\ (\Snd (\Snd y_2)))}}  \quad \text{// otherwise, return $t.$value$ + f(t.$left$) + f(t.$right$)$}\\
  &&  \ ( f_0 \ f_0) \ t \\
% \text{Type of } f_0 &=& \mu \alpha. \alpha \rightarrow \TR \rightarrow \RR \rightarrow \RR \\
\text{Transformation:} \\
\texttt{Transform}(f) &=& \lambda x. \ \Let{\hat x}{(x, \Ref 0)} \ (\lambda \tilde x. \ (\lambda k. \\
 && \quad \quad \Let{t}{\Right{((5, \Ref 0), (\Left{()},\Left{()}))}} \\
 && \quad \quad \Let{f_0}{\lambda f. \lambda k_1. \ k_1 \ (\lambda tt. \lambda k_2.\\
 && \quad \quad \Pat{tt}{y_1}{k_2 \ \tilde x}{y_2}{\\
 && \quad \quad \quad \quad (f \ f)(\lambda a_1. \ (a_1 \ (\Fst (\Snd y_2))) \ (\lambda a.\\
 && \quad \quad \quad \quad \Let{\hat z_1}{\Fst y_2} \\
 && \quad \quad \quad \quad \Let{\hat z_2}{a} \\
 && \quad \quad \quad \quad \Let{\hat z}{(z_1 + z_2, \Ref 0)} \\
 % && \quad \quad \quad \quad \quad  k_3 \ \hat z; \\
 && \quad \quad \quad \quad \quad (f \ f)(\lambda b_1. \ (b_1 \ (\Snd (\Snd y_2))) \ (\lambda b.\\
 && \quad \quad \quad \quad \quad \quad \Let{\hat w_1}{\hat z} \\
 && \quad \quad \quad \quad \quad \quad \Let{\hat w_2}{b} \\
 && \quad \quad \quad \quad \quad \quad \Let{\hat w}{(w_1 + w_2, \Ref 0)} \\
 && \quad \quad \quad \quad \quad \quad \quad k_2 \ \hat w; \\
 && \quad \quad \quad \quad \quad \quad \quad w_1' \mathrel{+}= \ ! \ w'; \\
 && \quad \quad \quad \quad \quad \quad \quad w_2' \mathrel{+}= \ ! \ w'))) \\
 && \quad \quad \quad \quad \quad z_1' \mathrel{+}= \ ! \ z'; \\
 && \quad \quad \quad \quad \quad z_2' \mathrel{+}= \ ! \ z')})} \\
 && \quad \quad (f_0 \ f_0) (\lambda a_{11}. (a_{11} \ t) \ k))) \\
 && \quad \quad \hat x \ (\lambda y. \Let{\hat y}{y} y' := 1.0); \\
 && \quad \quad ! \ x' \\
\ea
$$
\vspace{5ex}
% \clearpage
\bibliography{../references}

\end{document}
